
%%%%%%%%%%%%%%%%%%%%%%%%%%%%%%%%%%%%%%%%%%%%%%%%%%%%%%%%%%%%%%%%%%%%%%%%%%%%%%%%%%%%%%%%%%%%%%%%%

%%%%%%%%%%%%%%%%%%%%%%%%%%%%%%%%%%%%%%%%%%%%%%%%%%%%%%%%%%%%%%%%%%%%%%%%%%%%%%%%%%%%%%%%%%%%%%%%%
%%%%%%%%%%%%%%%%%%%%%%%%%%%%%%%%%%%%%%%%%%%%%%%%%%%%%%%%%%%%%%%%%%%%%%%%%%%%%%%%%%%%%%%%%%%%%%%%%

\section{Upper bound}
\label{sec:upper}

To show the optimality of \ALG, we prove that the competitive ratio of any
deterministic algorithm for the multiple knapsack problem is at most $\R -
O(1/n)$. This shows that the term $O(1/n)$ occurring in the competitive
ratio of our deterministic algorithm $\ALG$ is inevitable.

Our adversarial strategy is parameterized with a non-decreasing sequence of $n +
1$ numbers~$s(i)$, such that $1/2 < s(1) \leq s(2) \leq s(3) \leq \ldots \leq
s(n) \leq s(n+1) = 1$ and proceeds in at most $n+1$ phases numbered from $1$. In
phase $i$, the adversary issues $n$ items, all of size $s(i)$, and stops
immediately once \DET accepts one of these items (places it into a~bin). If all
$n$ items of a phase are rejected by \DET, the adversary finishes the input
sequence. 

Let $j$ be the number of phases where $\DET$ accepts an item. As the size of
each item from the input is strictly greater than $1/2$, at most one item fits
in a single bin, and thus $j \leq n$. Then, total load of \DET is equal to
$\DET(j) = \sum_{i = 1}^{j} s(i)$. Note that in the final phase $j+1$, $n$ items
of size $s(j+1)$ are presented (but rejected by \DET). \OPT may accept all of
them obtaining the total load $n \cdot s(j+1)$. By comparing these two gains and
as $\DET$ may choose the value of $j$, we get that the competitive ratio of
$\DET$ is at most
\begin{equation}
\label{eq:upper-bound-ratio}
U := \max_{0 \leq j \leq n} \frac{\DET(j) / n}{\OPT(j) / n} = 
  \max_{0 \leq j \leq n} \frac{ (1/n) \cdot \sum_{i = 1}^{j} s(i)}{s(j+1)}.
\end{equation}

It remains to choose an appropriate sequence $s$ and compute $U$ using \eqref{eq:upper-bound-ratio}. 
As a warm-up, we show a simple
construction leading to a weaker upper bound of $\R$, which we later refine to
obtain the bound of $\R - O(1/n)$. We note that the bound of $\R$ given by Cygan
et al.~\cite{CyJeSg16} uses concepts similar to ours, although in their
construction, the adversary tries to control the number of bins that store
particular amounts of items, while our approach is more fine-grained, and the
adversary controls the contents of particular bins. 

For our both bounds we use the following sequence $s$ with the function $\f$
defined in \eqref{eq:f-definition}, parameter $\alpha$ that we define later and
infinitesimally small but positive $\epsilon$. 
\begin{equation}
  \label{eq:ultimate_s}
    s(i) = \begin{cases}
      \f \left( \frac{i-1}{n} + \alpha \right) + \epsilon 
        & \text{if $i \in \{ 1, \ldots, n \}$}, \\
      1 & \text{if $i = n+1$}.
    \end{cases}
\end{equation}
The role of $\epsilon$ is to ensure that $s(i) > 1/2$ for all $i$. 
We will neglect the load caused by $\epsilon$ in our proofs below;
such approach could be formalized by taking $\epsilon = c/n^2$
with a constant $c$ tending to zero.


%%%%%%%%%%%%%%%%%%%%%%%%%%%%%%%%%%%%%%%%%%%%%%%%%%%%%%%%%%%%%%%%%%%%%%%%%%%%

\subsection{Upper bound of R}

\begin{lemma}
\label{lem:upper-bound-weak}
The competitive ratio of any deterministic algorithm \DET for the multiple
knapsack problem is at most $\R$.
\end{lemma}

\begin{proof}
The adversary uses the strategy described above
with sequence $s$ defined by \eqref{eq:ultimate_s} parameterized
with $\alpha = 0$.
We fix any $j \in \{0, \ldots, n\}$ and observe that $s(j+1) \geq \f(j/n)$ 
(for $j = n$ we use $\f(1) = 1$).
By \eqref{eq:upper-bound-ratio}, \eqref{eq:Ff} and \cref{lem:F-properties}, the competitive ratio of \DET is 
upper-bounded by 
\[
  U = \frac{ \sum_{i = 1}^{j} s(i)/n }{s(j+1)}
  = \frac{ \frac{1}{n} \cdot \sum_{i = 1}^{j} \f \left( \frac{i-1}{n} \right)}
    {s(j+1)}
  \leq \frac{ \int_0^{j/n} \f (x) \, \dd x }{s(j+1)}
  \leq \frac{\fintegral(j/n)}{\f(j/n)}
  \leq \R.
\qedhere
\]
\end{proof}



%%%%%%%%%%%%%%%%%%%%%%%%%%%%%%%%%%%%%%%%%%%%%%%%%%%%%%%%%%%%%%%%%%%%%%%%%%%%

\subsection{Upper bound of R - O(1/n)}

By choosing parameter $\alpha = 1/(8n)$, 
we may improve the upper bound by a term of $\Theta(1/n)$.
As in the proof of \cref{lem:upper-bound-weak}, 
we approximate the sum of $s(i)$'s by an appropriate integral, but 
this time we also bound the error due to such approximation.

\begin{lemma}
\label{lem:upper-bound-technical}
For the sequence $s$ defined in \eqref{eq:ultimate_s} parameterized
with $\alpha = 1/(8n)$, and for any $j \in \{0,\ldots,n\}$,
it holds that 
$\sum_{i=1}^j s(i)/n \leq \fintegral(j/n)$.
Moreover, for $n \geq 3$, it holds that 
$\sum_{i=1}^n s(i)/n \leq \fintegral(1) - 1/(52 n)$.
\end{lemma}

\begin{proof}
Note that $\sum_{i = 1}^j s(i)/n
= \sum_{i = 0}^{j-1} s(i+1) / n
 = \sum_{i = 0}^{j-1} \frac{1}{n} \cdot f(i/n+\alpha )$.
On the other hand, 
$F(j/n) = \int_0^{j/n} f(x) \,\dd x = \sum_{i = 0}^{j-1} \int_{0}^{1/n} \f(i/n + x) \,\dd x$.
Therefore, $F(j/n) - \sum_{i = 1}^j s(i)/n = \sum_{i=0}^{j-1} D_i$, where 
\begin{equation}
\label{eq:def_D}
  D_i = \int_{0}^{1/n} \f\left( i/n + x \right) 
      - \f\left( i/n + \alpha \right) \,\dd x .
\end{equation}
Now we estimate $D_i$. We consider two cases.
\begin{enumerate}
\item  $i/n + \alpha < \R$. Then, $f(i/n+\alpha) = 1/2$ and 
$\f(i/n+x) \geq 1/2$ for any $x \in [0,1/n]$. Substituting these bounds 
to \eqref{eq:def_D} yields $D_i \geq 0$. 

\item $i/n + \alpha \geq R$.
In this case, $\f(i/n + \alpha) = (2 \e)^{i/n + \alpha - 1}$ and 
$\f(i/n + x) \geq (2 \e)^{i/n + x - 1}$ for any $x \in [0,1/n]$.
Thus, 
\begin{align*}
    D_i 
      & \geq \int_{0}^{1/n} (2 \e)^{i/n + x - 1} - (2 \e)^{i/n + \alpha - 1} \,\dd x 
       = (2 \e)^{i/n - 1} \cdot 
        \int_{0}^{1/n} (2 \e)^x - (2 \e)^\alpha \,\dd x .
\end{align*}
%
By the Taylor expansion of function $(2\e)^x$,
for any $\beta \in [0,\R]$ it holds that
$1 + \beta/R \leq (2 \e)^\beta \leq 1 + 2 \cdot \beta/\R$. Thus, 
%
\begin{align*}
    D_i & \geq (2 \e)^{i/n - 1} \cdot 
        \int_{0}^{1/n} \frac{x}{\R} - \frac{2 \alpha}{\R} \,\dd x 
      \geq (2 \e)^{i/n - 1} \cdot \left( \frac{1/n^2}{2 \R} - \frac{2 \alpha / n}{\R} \right) \\
      & \geq \frac{1}{2 \e} \cdot \frac{1}{4 \R \cdot n^2} 
      > \frac{1}{13 \cdot n^2}.
\end{align*}
As $D_i \geq 0$ in either case, we immediately get the first part
of the lemma. 
To show the second part, we set $j = n$ and 
observe that all integers $i \in \{\lceil n \cdot \R \rceil, \ldots, j-1\}$ 
are considered in the second case above.
Therefore $F(1) - \sum_{i=1}^n s(i)/n = \sum_{i=0}^{n-1} D_i \geq 
(n-\lceil n \cdot R \rceil)/ (13 \cdot n^2)$. 
For $n \geq 3$, it holds that $n-\lceil n \cdot R \rceil \geq n/4$, which
concludes the proof.
\qedhere
\end{enumerate}
\end{proof}



\begin{theorem}
\label{thm:upper-bound}
The competitive ratio of any deterministic algorithm \DET for the multiple knapsack problem 
is at most $\R - 1/(52 n)$.
\end{theorem}

\begin{proof}
The adversary uses the strategy described above
with sequence $s$ defined by \eqref{eq:ultimate_s} parameterized
with $\alpha = 1/(8n)$. We compute the upper bound $U$ on the competitive ratio
of \DET using \eqref{eq:upper-bound-ratio}.

We separately consider the case of small value of $n$.
If $n = 1$, then $s(1) = 1/2$ and $s(2) = 1$, and therefore 
$U = 1/2$. If $n = 2$, then $s(1) = s(2) = 1/2$ and $s(3) = 1$,
and thus again $U = 1/2$. In these cases, $U = 1/2 < R - 1/(52n)$, and 
the theorem follows. 

Below, we assume that $n \geq 3$.
We fix any $j \in \{0, \ldots, n\}$ and let $y = j/n$.
We consider two cases. 

\begin{enumerate}
\item $j < n$. 
Then, by \cref{lem:upper-bound-technical}, $\sum_{i=1}^j s(i)/n \leq \fintegral(y)$. 
Additionally, $s(j+1) = \f(y+\alpha) + \epsilon > \f(y + \alpha)$. 
Using \cref{lem:F-properties}, we obtain
\begin{equation}
\label{eq:ub-1}
U = \frac{ (1/n) \cdot \sum_{i = 1}^{j} s(i)}{s(j+1)}
  < \frac{\fintegral(y)}{\f(y + \alpha)} 
  = \frac{\fintegral(y)}{\f(y)} \cdot \frac{\f(y)}{\f(y + \alpha)} 
  = \min\{y, \R\} \cdot \frac{\f(y)}{\f(y + \alpha)}.
\end{equation}
It suffices to show that $U \leq R - \alpha/2 = R - 1/(16n)$. 
\begin{enumerate}
\item If $y < \R - \alpha/2$, then $U < y \cdot 1 < \R - \alpha/2$.
\item If $y \geq \R - \alpha/2$, then we upper-bound the fraction
$\f(y) \,/\, \f(y + \alpha)$. As $y + \alpha/2 \geq \R$,
\begin{equation}
\label{eq:ub-2}
  \frac{\f(y)}{\f(y+\alpha)} 
  \leq \frac{\f(y+\alpha/2)}{\f(y+\alpha)} 
  = \frac{(2\e)^{y+\alpha/2-1}}{(2\e)^{y+\alpha-1}} 
  = (2\e)^{-\alpha/2}
  \leq 1-\frac{\alpha}{2\R}.
\end{equation}
In the last inequality, we used the Taylor expansion of the function $(2\e)^x$.
Combining \eqref{eq:ub-1} with \eqref{eq:ub-2} yields 
$U \leq R - \alpha/2$ as desired.
\end{enumerate}

\item $j = n$.
By \cref{lem:upper-bound-technical}, 
$\sum_{i=1}^n s(i)/n \leq R - 1/(52n)$ and $s(n+1) = 1$.
Therefore, in this case, $U \leq R - 1/(52n)$.
\qedhere
\end{enumerate}
\end{proof}

